
\usepackage[english]{babel}


%\bibliographystyle{IEEEtran}

%\usepackage[square]{natbib}
\usepackage{graphicx}
\usepackage{framed}
\usepackage{multirow}
\usepackage{lipsum}  
\usepackage{verbatim}
\usepackage{amsmath}
\usepackage{longtable}
\usepackage{caption}
\usepackage{subcaption}
\usepackage{varwidth}

\usepackage{array}
\usepackage{adjustbox}


%\usepackage[oneside,width=17.5cm,height=24cm,left=2cm]{geometry}
%\usepackage[nolist,nohyperlinks]{acronym}

\usepackage[]{nomencl}
\usepackage[final]{pdfpages}

%\usepackage{biblatex}
%\addbibresource{references.bib}
\usepackage{mdframed}


\newcolumntype{L}[1]{>{\raggedright\let\newline\\\arraybackslash\hspace{0pt}}m{#1}}
\newcolumntype{C}[1]{>{\centering\let\newline\\\arraybackslash\hspace{0pt}}m{#1}}
\newcolumntype{R}[1]{>{\raggedleft\let\newline\\\arraybackslash\hspace{0pt}}m{#1}}

%%%  para ter capítulos xpto
%%%  https://hstuart.dk/2007/05/21/styling-the-chapter/


\usepackage{tikz, blindtext}
\usepackage{kpfonts}
\usepackage[explicit]{titlesec}
\usepackage{xcolor}
\definecolor{FEUP_color}{RGB}{140,45,25}

\newcommand*\chapterlabel{}
\titleformat{\chapter}
{\gdef\chapterlabel{}
	\normalfont\sffamily\huge\bfseries\scshape}
{\gdef\chapterlabel{\thechapter\ }}{0pt}
{\begin{tikzpicture}[remember picture,overlay]
	\node[yshift=-6cm] at (current page.north west)
	{\begin{tikzpicture}[remember picture, overlay]
		%\draw[fill=white] (-1,0) rectangle
	%	(1.2\paperwidth,8cm);
		\node[anchor=west,xshift=3cm,rectangle,
		rounded corners=1pt,inner sep=11pt,
		fill=black]
		{\color{white}\chapterlabel#1};
		\end{tikzpicture}
	};
\end{tikzpicture}
}
\titlespacing*{\chapter}{0pt}{50pt}{50pt}

\makenomenclature
\chapter*{Acronyms}
%\addcontentsline{toc}{chapter}{Acronyms}

%----------------------
%Acronyms List 
%----------------------
{
	\footnotesize
\begin{flushleft}
	\begin{tabular}{l p{0.8\linewidth}}
		
		\\
		
		AC	&	Alternating Current	\\
		AMR	&	Automatic Meter-Reading system	\\
		CEBD	&	Compiled Energy Billing Data-sets	\\
		DC	&	Direct Current	\\
		DCS	&	Data Collecting System	\\
		DHS	&	Data Handling System	\\
		DSS	&	Decision Support System	\\
		DSSS	&	Direct Sequence Spread Spectrum	\\
		FEM	&	Finit Element Method	\\
		EMI	&	Electromagnetic Interference	\\
		EETC	&	Energy-efficient Train Control	\\
		EETT	&	Energy-Efficient Train Timetabling	\\
		EMF	&	Energy Measurement Function	\\
		EMS	&	Energy Measurement System	\\
		ERA	&	European Union Agency for Railways	\\
		EU	&	European Union	\\
		GMSK	&	Gaussian Minimum Shift Keying	\\
		GPS	&	Global Position System	\\
		GSM	&	Global System for Mobile communications	\\
		GTO	&	Gate Turn-off Thyristors	\\
		ICT	&	Information and Communication Technology	\\
		IGBT	&	Insulated Gate Bipolar Transistors	\\
		IP	&	Internet Protocol	\\
		IP3	&	Innovation Programme 3	\\
		ISM	&	Industrial, Scientific and Medical	\\
		KPI	&	Key Performance Indicators	\\
		LAN	&	Local Area Network	\\
		LC	&	Inductor-Capacitor	\\
		LTE	&	Long-Term Evolution	\\
		MAC	&	Medium Access Control	\\
		MDMS	&	Meter Data Management System	\\
		OFDM	&	Orthogonal Frequency Division Multiplexing	\\
		OSI	&	Open Systems Interconnection	\\
		PLC	&	Power Line Communication	\\
		QoS	&	Quality of Service	\\
		PHY	&	Physical Layer	\\
		RTS	&	Railway Transportation System	\\
		RUs	&	Railway Undertakings	\\
		S2R	&	Shift2Rail	\\
		SG	&	Smart Grid	\\
		SMD	&	Smart Metering Demonstrator	\\
		TCMS	&	Train Communication \& Management System	\\
		TSIs	&	Technical Specifications for Interoperability	\\
		WLAN	&	Wireless LAN	\\
		WSN	&	Wireless Sensor Networks	\\

		
		
		
		
	\end{tabular}
\end{flushleft}
}
%----------------------
% Automatic Acronyms Generator
%----------------------
\acrodef{}{}

\acrodef{AC}{Alternating Current}
\acrodef{AMR}{Automatic Meter-Reading system}
\acrodef{CEBD}{Compiled Energy Billing Data-sets}
\acrodef{DC}{Direct Current}
\acrodef{DCS}{Data Collecting System}
\acrodef{DHS}{Data Handling System}
\acrodef{DSS}{Decision Support System}
\acrodef{DSSS}{Direct Sequence Spread Spectrum}

\acrodef{FEM}{Finit Element Method}
\acrodef{EMI}{Electromagnetic Interference}
\acrodef{EETC}{Energy-efficient Train Control}
\acrodef{EETT}{Energy-Efficient Train Timetabling}
\acrodef{EMF}{Energy Measurement Function}
\acrodef{EMS}{Energy Measurement System}
\acrodef{ERA}{European Union Agency for Railways}
\acrodef{EU}{European Union}
\acrodef{GMSK}{Gaussian Minimum Shift Keying}
\acrodef{GPS}{Global Position System}
\acrodef{GSM}{Global System for Mobile communications}
\acrodef{GTO}{Gate Turn-off Thyristors}
\acrodef{ICT}{Information and Communication Technology}
\acrodef{IGBT}{Insulated Gate Bipolar Transistors}
\acrodef{IP}{Internet Protocol}
\acrodef{IP3}{Innovation Programme 3}
\acrodef{ISM}{Industrial, Scientific and Medical}

\acrodef{KPI}{Key Performance Indicators}
\acrodef{LAN}{Local Area Network}
%\acrodef{LC}{Inductor-Capacitor}
\acrodef{LTE}{Long-Term Evolution}
\acrodef{MAC}{Medium Access Control}
\acrodef{MDMS}{Meter Data Management System}
\acrodef{OFDM}{Orthogonal Frequency Division Multiplexing}
\acrodef{OSI}{Open Systems Interconnection}
\acrodef{PLC}{Power Line Communication}
\acrodef{QoS}{Quality of Service}
\acrodef{PHY}{Physical Layer}

\acrodef{}{}
\acrodef{RTS}{Railway Transportation System}
\acrodef{RUs}{Railway Undertakings}
\acrodef{S2R}{Shift2Rail}
\acrodef{SG}{Smart Grid}
\acrodef{SMD}{Smart Metering Demonstrator}
\acrodef{TCMS}{Train Communication \& Management System}
\acrodef{TSIs}{Technical Specifications for Interoperability}
\acrodef{WLAN}{Wireless LAN}
\acrodef{WSN}{Wireless Sensor Networks}





