




\section{Objectives}

This work is focused on measuring the energy flows in the railway system.
The aim of this work is to improve the energy efficiency in the railway transportation system (RTS) and reduce the maintenance cost of RTS power systems.
The implementation of smart meters (SM) in RTS promote a better overview of power flow and, based on the information of SM, algorithms focusing on energy efficiency can be implemented.
The SM requires sensors such as voltage and current sensors. The level of intrusion as well as the level of electric <<valor da grandeza electrica>> of such sensors implies considerable costs of the sensors.
Therefore, the implementation of complex processing on smart meters is of added value. This complex processing can be the implementation of fault monitoring algorithms in SM based on the energy measurements.
Framed in the shif2rail, the work is focused on the implementation of a smart meter demonstrator for the RTS. To embrace the entire railway system, the power flow should consider the energy flux from and to the catenary. Therefore, the key point should be the measurement of the energy in the traction substations and in the train power transformer 
Based on this thesis proposal, the objectives are the following:

\begin{itemize}
	\setlength\itemsep{0em}
	
	\item	Modeling, development and implementation of a metering system based on a non-intrusive self-powered sensor node.

	\item Modeling and simulation of a RTS wireless network.

	
\end{itemize}


\section{Contributions}

\begin{itemize}
	\setlength\itemsep{0em}
	
	\item Availability of measured data from trains where currently no energy measurement is performed.
	
	\item Data-rate increase of energy measurements, which will result on direct increase on the quality of information of energy.
	
	\item A further contribution can be the avoidance of broadband real-time/continuous communication (such as LTE), being possible to collect and store data in train data concentrator and, while the train is waiting at stations, the data is transferred between train and station AP (and then to a remote server). A possible contribution will be the cost reduction of information transmission of energy sensor network data.
	
	\item New advanced sensor node for AC current measurement;
	
	\item Possible contribution: Given the measurement characteristics, a self powered wireless sensor node can implement features of high processing.
	
\end{itemize}





