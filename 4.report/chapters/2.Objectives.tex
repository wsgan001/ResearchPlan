

\lipsum[4-4]


\section{Objectives}

This work is focused on measuring the energy flows in the railway system.
The aim of this work is to improve the energy efficiency in the railway transportation system (RTS) and reduce the maintenance cost of RTS power systems.
The implementation of smart meters (SM) in RTS promote a better overview of power flow and, based on the information of SM, algorithms focusing on energy efficiency can be implemented.
The SM requires sensors such as voltage and current sensors. The level of intrusion as well as the level of electric <<valor da grandeza electrica>> of such sensors implies considerable costs of the sensors.
Therefore, the implementation of complex processing on smart meters is of added value. This complex processing can be the implementation of fault monitoring algorithms in SM based on the energy measurements.
Framed in the shif2rail, the work is focused on the implementation of a smart meter demonstrator for the RTS. To embrace the entire railway system, the power flow should consider the energy flux from and to the catenary. Therefore, the key point should be the measurement of the energy in the traction substations and in the train power transformer 
Based on this thesis proposal, the objectives are the following:

1.	Research on high-voltage and high-current measurement systems

2.	Research of train power transformer and implementation of a simulation model of a train power transformer.

3.	Development and implementation of a measuring system with high acquisition and processing capabilities.

4.	Research on communication systems and development of a network model in a simulation environment.

5.	Research, development and implementation of a fault monitoring system.

6.	Research, development and implementation of an energy flow monitoring system.

7.	Implementation and validation of SM in a pilot project through real tests.


\section{Contributions}

1.	Increase of energy flow information of RTS

2.	Reduction of transmission costs of information (no need of LTE, the data are concentrated and transmitted from trains to stations, during passenger exchange, with a high throughput link)

3.	Decrease of the Life Cycle Cost (LCC) of 




