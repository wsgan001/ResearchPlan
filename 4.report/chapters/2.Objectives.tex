

\section{Objectives}

Nowadays, the need for increasing the energy efficiency in the \ac{RTS} is a relevant topic.
We can see two ways to increase energy efficiency: (1) we can act in the construction of new trains, using the most energy efficiency technologies or (2) we can adapt the current operation of \ac{RTS} in order to promote a better use of the available resources.
On a higher level, this work will contribute to increase the energy efficiency of \ac{RTS} operation, thus optimizing the resources available.

Currently, there is a lack of knowledge on the \ac{RTS} energy flow, resulting in a poor operation of this system. 
%The decisions taken in the train operation have lack of information from field infrastructure, in both the existence of such information and in terms of quality of the information.
Information from field infrastructure is not taken into account when a decision is made in the train operation, either because there is a lack of information or because of the reduced quality of information.
 Presently we have trains that do not have available, for the operation point of view, the information on power flow. In recent years, \ac{EU} has promoted the implementation of automatic metering systems in train operations, mainly for billing purposes. However, even with this \ac{EU} regulation, the periodicity of available data is around 5 minutes, putting in question the quality of the information.

%To have the knowledge necessary to take actions towards the 
It seems logical that to promote a better operation of the \ac{RTS}, the information of energy usage from the field must be used.
This work is focused on collecting information from the field of operation and make it available, in a centralized database (for knowledge extraction mechanisms). 
To collect the information, we have to perform measurements of the energy related variables.

Thereby, the objectives can be synthesized in the following two points:

%This work is focused on measuring the energy flows in the railway system.
%The aim of this work is to improve the energy efficiency in the railway transportation system (RTS) and reduce the maintenance cost of RTS power systems.
%The implementation of smart meters (SM) in RTS promote a better overview of power flow and, based on the information of SM, algorithms focusing on energy efficiency can be implemented.
%The SM requires sensors such as voltage and current sensors. The level of intrusion as well as the level of electric <<valor da grandeza electrica>> of such sensors implies considerable costs of the sensors.
%Therefore, the implementation of complex processing on smart meters is of added value. This complex processing can be the implementation of fault monitoring algorithms in SM based on the energy measurements.
%Framed in the shif2rail, the work is focused on the implementation of a smart meter demonstrator for the RTS. To embrace the entire railway system, the power flow should consider the energy flux from and to the catenary. Therefore, the key point should be the measurement of the energy in the traction substations and in the train power transformer 
%Based on this thesis proposal, the objectives are the following:

\begin{itemize}
	\setlength\itemsep{0em}
	
	\item	Research on \textbf{railway energy models}, and \textbf{development/implementation of a metering system} for railway power flow monitoring.
	This is expected to be based on a non-intrusive self-powered sensor node inserted into train power system.

	\item Research on \textbf{communication network models} for a \ac{RTS} wireless network with \textbf{validation through simulation frameworks}.
	\textbf{Development and implementation} of \ac{RTS} wireless network to store the energy information data of railway into central database.

	
\end{itemize}

\newpage
\section{Contributions}

The expected contributions will be divided in two major areas: (1) the energy measurement and information generation of the sensor nodes and (2) the energy information transmission and storage into centralized database.

On the energy measurement and information generation of the sensor nodes, two contributions can be considered as follows:

\begin{itemize}
	\setlength\itemsep{0em}
	
	\item \textbf{New energy metering architecture}, according to some specifications such as the usage of a non-intrusive approach.
	This architecture will generate energy information about the power flow of the railway system.
	
	\item \textbf{Accurate estimation of power flow} into catenary, based on on-board measurements. The available parameters will be: (1) the RMS voltage, current and apparent power, (2) the instantaneous active power, reactive power, power factor and frequency, and (3) the cumulative energy consumptions in terms of kVAh, kVARh and KWh.
	
	
\end{itemize}


Regarding the energy information transmission and storage into centralized database, the following contributions are expected:


\begin{itemize}
	\setlength\itemsep{0em}
	
	\item \textbf{Availability of measured data} from trains where currently limited/inexistent energy measurement is performed.
	
	\item Data-rate increase of energy measurements, which will result onin direct \textbf{increase on the quality of information of energy}. This increase will overcome the 5-minute data-rate that currently are used in energy meters.

	
	
\end{itemize}

A further contribution can be the avoidance of broadband real-time/continuous communication (such as \ac{LTE}), with the direct cost reduction of information transmission of energy \ac{RTS} data.

