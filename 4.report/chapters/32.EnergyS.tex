


\section{Section 2: Energy sensors}

1.	Sensor overview – historic perspective

2.	Current transducers and voltage transducers
a.	Commonly used technologies and principles
b.	New breakthroughs

3.	High power measurement challenges in RTS

4.	Energy measurement technologies in RTS 


\subsection{Electric Energy Overview}

Energy in form of electricity is the major traction player on the RTS. 
Complementary to the Diesel, this form of energy is distributed along with the rails in catenaries.
As presented in previous section, there are two major ways of rail electrification: AC or DC.

When a train is connected to a DC transmission line, it is possible to exchange energy from and to the catenary, complying to the khirshoffs law. The catenary is considered as a node that has multiple bi-directional power flow elements, specifically, trains and traction substations.

On AC transmission lines, the power flow is directly related to the relation between the voltage and current waveforms. 

\subsection{Current transducers and voltage transducers}
	
a.	Commonly used technologies and principles
b.	New breakthroughs

\subsection{High power measurement challenges in RTS}	

\subsection{Energy measurement technologies in RTS }