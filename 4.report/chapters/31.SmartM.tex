


\section{Section 1: Smart metering}

In this section, special attention is given to smart metering. The smart grid framework is presented to justify the need for smart meters and then, an overview on metering systems is presented.


1.	Smart grids and the need for smart meters

2.	Metering systems – overview

3.	Metering systems in railways

4.	Wireless sensor networks – overview

5.	Smart metering with WSN in Railway Transportation System (RTS)

\subsection{Smart grids and the need for smart meters}

Smart Grids improves the functionality and concept of traditional electrical grids by obtaining the grid component's data using Information and Communication Technology (ICT). Such grids benefit the reliability and the efficiency of the system with the usage of the acquired data, \cite{Mohassel2014}.

Although the smart meters does not have an effective definition, those devices are composed by an electronic box and a communication link, \cite{Seppo2012}. A smart meter is responsible for measuring the energy-related parameters and the user consumption with a given time interval. All those measurements are then transmitted upon a communication network to the utility or to other player with the responsibility of using the meter data. The information obtained from the data is shared with consumer-side devices, to inform the end-users on their related costs and energy usage, \cite{Siano2014}.

Smart meters implement a bidirectional communication on top of AMR. They are inherent to smart grid systems. 

Similarly to the evolution of electricity meters, the utility grid has evolved from a centralized production and control perspective to a distributed one. The conventional electrical grid is a network with a transmission link connecting power producers and end-user consumers. The control and distribution of electrical power is made in a centralized way. With the increase of power demand, increase of complexity and having more and more decentralized power generation, a migration to the smart grid framework is required, \cite{Reddy2014}.

\subsection{Features of Smart Meters and metering systems}

A smart metering system combines several controlling devices, a extensive number of sensors for measuring the parameters and devices responsible for the transferring of the data and the commands. The detection of unauthorized consumption due to electrical energy theft and the improvement of the energy in the distribution are other advantages of smart meters. These devices acts as a gateway by having a communication interface protocol to the database stored by the utility company, \cite{Reddy2014}.

The design of an ideal smart grid has to focus on prediction, adaptability and reliability points. Moreover, it requires to cover the demand adjustment, the load handling, flexibility and sustainability and it should incorporate advanced services. In advance, an end to end control capability has to be ensured as well as finding the optimal cost and asses, increase the quality of energy and quality of service. Another features of smart grids are the automatic restoration ans self-healing, being all the previously presented features of the smart grids highly dependent of the role of the smart meters, \cite{Mohassel2014}.

Smart-meter types are also distinguished based on features like data-storage, communication type and connection with the energy supplier. The data storage capability allows data to be stored in the meter, being transferred after a few days or weeks to the Meter Data Management System (MDMS) of the utility. Compensations for some power quality deficiencies can be also considered; therefore the future meters should be also capable of register certain basic power quality characteristics. In advance, the design of rate and tariffs of electricity providers determine the requirements such as the period of meter intervals or the temporal resolution (commonly ranging from 15 min to 1 h). During those intervals, the production and consumption of active and reactive power is mandatory to be separately measured, \cite{Siano2014}.

\subsection{Metering systems in railways}


\subsection{Wireless sensor networks – overview}


\subsection{Smart metering with WSN in Railway Transportation System (RTS)}


%%%%%%%%%%%%%%%%%%%%%%%%%%%%%%%%
%%%%%%%%%%%%%%%%%%%%%%%%%%%%%%%%%
%%%%%%%%%%%%%%%%%%%%%%%%%%%%%%%%%