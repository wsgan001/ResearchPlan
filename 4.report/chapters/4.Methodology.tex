\lipsum[4-4]

Note: not in chronological order

\section{RTS wireless network}





1.	Purpose: model and simulate a WSN for energy measurement of RTS rolling stock, with an advanced network infrastructure (englobing both train WSN and station AP’s)

2.	Contribution: An energy measurement system in rolling stock does not require a broadband real-time/continuous communication (such as LTE), being possible to collect and store data in train data concentrator and, while the train is waiting at station for passenger exchange (which lasts for less than one minute), the data is transferred between train and station AP (and then to a remote server). Therefore, the contribution will be the cost reduction of information transmission of energy sensor network data

3.	Methodology:
a.	Modeling of energy sensor network of rolling stock: sensor nodes and data concentrator
b.	Modeling of infrastructure: train concentrators, station AP, station data “buffer” and station internet connection
c.	Implementation in simulation environment of such models, using NS3 simulator or similar
d.	Definition of “sensor data rate” as function of the line length-between-stations ()


\section{Non-intrusive self-powered sensor node}

1.	Purpose: In the scope of Shift2Rail, is expected to develop a smart meter for railways. The purpose is to model, simulate and implement a series of sensor nodes for current measurement in the transformer´s secondary windings. Assuming that the railway environment requires non-intrusive measurement devices and, if possible, self-powered, a set of requirements is then identified for the sensor node:
a.	Electrically non-intrusive (using hall-effect, rogowsky or current transformer principles; without the need for mechanically changing the windings)
b.	Self powered, if the current transformer has sufficient power capabilities
c.	With high processing capabilities, high acquisition frequency and sufficient amount of memory 
i.	Variable acquisition in tens of samples per second (according to the power quality standard of 15kHz <?>)
ii.	Frequency analysis capability
iii.	Capable of implement outlier detection algorithms 

2.	Contribution: new advanced sensor node for high current measurement
3.	Other Contribution: given the measurement characteristics, a self powered wireless sensor node can implement features of high processing.
4.	Methodology to be defined

\section{Rolling stock traction transformer model}


1.	Purpose: model the train transformer with two perspectives:
a.	Efficiency estimation based on secondary measurements
b.	Evaluation of transformer operation towards fault detection
2.	Possible contribution: an accurate model for train transformer, capable of efficient estimation of energy consumption based on secondary windings current measurements
3.	Possible contribution: assuming that the influence of transformer in the power life cycle cost is relevant (see note), the contribution will be the operation monitoring towards maintenance cost reduction.
4.	Methodology:
a.	Study failure rates of trains/transformers
b.	Model in a simulation environment the power transformer
c.	Identify and model transformer failures
d.	Implement in sensor nodes an energy estimation mechanism based on the loss model of the transformer and sensor nodes measurements
e.	Implement in sensor nodes a frequency analysis towards operation monitoring
f.	Prepare and implement results in field operation
